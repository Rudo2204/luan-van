\chapter{Kết luận}
\section{Đánh giá kết quả}
Nói chung, em đã có một trải nghiệm mới và tốt khi thực hiện một hệ thống nhúng sử dụng Rust trong suốt quá trình thực hiện đề tài này.
Trong môi trường nhúng tuy ta không thể sử dụng được thư viện std của Rust gồm rất nhiều tính năng mạnh mẽ, chất lượng, v.v.. nhưng kể cả bỏ qua thư viện chuẩn, sử dụng thư viện core của Rust, cùng với hệ thống ownership mới mẻ đã tạo nên một bộ công cụ hỗ trợ công việc thực hiện một hệ thống nhúng một cách dễ dàng và hiệu quả.

Hệ sinh thái nhúng của Rust được xây dựng một cách hiệu quả, an toàn ngay từ gốc đã góp phần không nhỏ trong việc sử dụng ngôn ngữ này trong hệ thống nhúng một cách hiệu quả và dễ dàng hơn. Ngoài ra, nó cũng góp phần tạo nên tính portability rất cao cho mã nguồn Rust. Hơn nữa, cũng vì tính portability cao này mà ta có thể dễ dàng học được cách viết các mã nguồn chất lượng cao từ các crate trong hệ sinh thái của Rust.

Tuy hệ sinh thái nhúng của Rust còn nhiều thiếu sót nhưng trong những năm tiếp theo khi Rust nhận được nhiều sự chú ý từ các tập đoàn, công ty chuyên về lập trình nhúng thì trong tương lai các thiếu sót này sẽ dần được bổ sung, hoàn thiện. Trong tương lai gần, có thể hệ sinh thái nhúng của Rust sẽ phát triển đủ mạnh để dần chiếm được thị trường nhúng, nơi mà C đang làm chủ ở mọi mặt.

Thông qua việc thực hiện năm ví dụ có độ khó từ đơn giản đến tương đối phức tạp ở phần trên, có thể khẳng định rằng Rust hoàn toàn có thể được sử dụng để thực hiện một hệ thống nhúng trong thực tế trong hôm nay một cách dễ dàng và thuận tiện.
Các hệ thống này được nhận các lợi ích từ hệ thống ownership system, giúp chúng tránh được nhiều lỗi thường gặp trong hệ thống C ngay trong lúc biên dịch chương trình.

Mã nguồn của hệ thống có thể được quản lý, debug, test, v.v.. một cách dễ dàng sử dụng công cụ \mintinline{bash}{cargo} đi kèm với ngôn ngữ.
Các hệ thống cũng được kiểm tra qua Github CI một cách thuận tiện, không gặp trở ngại.
Các công cụ khác hỗ trợ cho việc thiết kế và thực hiện ngôn ngữ Rust như IntelliJ hay Esclipse RustDT đang được phát triển một cách nhanh chóng.
Trong tương lai gần, các bộ công cụ này có thể trở thành cánh tay phải đắc lực cho các nhà phát triển ngôn ngữ Rust.

Một điểm yếu lớn trong quá trình thực hiện đề tài mà em rút ra được là: vì Rust là một ngôn ngữ còn mới, vì vậy tài liệu học tập cũng như các đề tài thực hiện sử dụng ngôn ngữ này là tương đối hạn chế.
Một người lập trình muốn thử nghiệm ngôn ngữ Rust có thể sẽ gặp phải nhiều vướng mắt trong quá trình thực hiện.
Tuy nhiên, cộng đồng những người lập trình Rust rất ưu ái đối với những câu hỏi về ngôn ngữ, cũng như cách thực hiện viết các mã nguồn chất lượng cao đã giúp em tự tin hơn trong quá trình thực hiện đề tài này.

\section{Hướng phát triển}
Rust là một ngôn ngữ mới nhưng có tiềm năng phát triển rất lớn, kể cả trong mảng lập trình nhúng.
Để góp phần thúc đẩy sự phát triển này, em đề xuất một số hướng như sau:
\begin{enumerate}
    \item Mở rộng tầm ảnh hưởng cũng như nhận thức của mọi người về ngôn ngữ mới này:
        Rust càng nhận được nhiều sự quan tâm từ mọi người, cũng như các công ty, tập đoàn chuyên về lập trình nhúng thì nguồn lực đổ vào để phát triển ngôn ngữ này sẽ càng lớn hơn.
    \item Thúc đẩy sự phát triển của các bộ công cụ hỗ trợ cho việc thiết kế một hệ thống Rust:
        Một bộ công cụ hỗ trợ tốt chính là chìa khóa để có được trải nghiệm tốt nhất khi thử nghiệm một ngôn ngữ mới.
    \item Hỗ trợ, bổ sung thêm các crate chất lượng cao:
        Hệ sinh thái nhúng của Rust còn tương đối non nớt vì vậyviệc bổ sung thêm các crate là điều cực kỳ cần thiết.
    \item Hỗ trợ viết thêm các tài liệu học tập cho Rust:
        Như đã phân tích ở phần trên, tài liệu học tập Rust, đặc biệt là trong mảng lập trình nhúng là tương đối hạn chế.
        Vì vậy, hỗ trợ viết các tài liệu học tập này sẽ góp phần không nhỏ trong việc thúc đẩy sự phát triển của Rust.
\end{enumerate}
